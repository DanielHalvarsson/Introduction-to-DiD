
\documentclass[usenames,dvipsnames,10pt]{beamer} 
% Add option 'aspectratio=169' for 16:9 widescreen 
% Add option  'handout' to ignore animations
% If you have a smaller amount of text, feel free to also try '11pt'! / Jesper
%\usepackage{listings}
\usepackage[flushleft]{threeparttable}
\usepackage{booktabs}
\usepackage[utf8]{inputenc}
\usepackage{verbatim}

%%% Bibliography
\usepackage[style=authoryear,backend=biber]{biblatex}
\addbibresource{bibliography.bib}



%%% Suppress biblatex annoying warning
\usepackage{silence}
\WarningFilter{biblatex}{Patching footnotes failed}

%%% Some useful commands
% pdf-friendly newline in links
\newcommand{\pdfnewline}{\texorpdfstring{\newline}{ }} 
% Fill the vertical space in a slide (to put text at the bottom)
\newcommand{\framefill}{\vskip0pt plus 1filll}
\newcommand{\sym}[1]{\rlap{#1}} % Thanks to Joseph Wright & David Carlisle

%%% Additional packages, added by Jesper Erixon
% Use babel to neatly translate 'abstract' etc. to swedish  or other supported language
%\usepackage[swedish]{babel}

%%% Enter additional packages below (or above, I can't stop you)! / Jesper
\renewcommand{\proofname}{\sffamily{Proof}}

%%%%%%%%%%%%%%%%%%%%%%%%%%%%%%%%%%%%%%%%%%%%%%%%%%%%%%%%%%%%%%%%%%%%%%%%%%%%%%%%%%%%%
%%%%%%%%%%%%%%%%%%%%%%%%%%%%%%% YOUR PRESENTATION BELOW %%%%%%%%%%%%%%%%%%%%%%%%%%%%%
%%%%%%%%%%%%%%%%%%%%%%%%%%%%%%%%%%%%%%%%%%%%%%%%%%%%%%%%%%%%%%%%%%%%%%%%%%%%%%%%%%%%%

\begin{document}

\begin{frame} 
\frametitle{Comments to JAQ of All Trades: Job Mismatch,
Firm Productivity and Managerial Quality}
\framesubtitle{By Luca Corragio, Marco Pagano, Annalisa
Scognamiglio and Joacim Tåg}

\begin{itemize}
    \item JAQ -  Job-worker Allocation Quality
    \item "Jack of all trades, master of none" is a figure of speech used in reference to a person who has dabbled in many skills, rather than gaining expertise by focusing on only one (from wiki)
\end{itemize}
\end{frame}

\begin{frame} 
\frametitle{A framework} 
\begin{exampleblock}{Productivity and wages}
... "insofar as an improvement in job allocation generates productivity gains, these are likely to be partly appropriated by workers in the form of higher wages"
\end{exampleblock}
\begin{itemize}
    \item  Seemingly related literature on rent-sharing, connecting firm productivity to wages.
   \end{itemize}
\begin{itemize}
    \item Value added per worker: $\frac{VA_j}{N_j} = \overbrace{P_J T_j}^{\text{TFP}_j}\frac{F(N_j,K_j)}{N_j}$        
   \end{itemize}
\begin{itemize}
    \item Quasi rent per worker: $\frac{Q_j}{N_j} = \frac{VA_j}{N_i} - w^a$
   \end{itemize}
\begin{itemize}
    \item Wages: $w_i = w^a + \gamma\frac{Q_j}{N_j}$
\end{itemize}    
\end{frame}

\begin{frame} 
\frametitle{Managerial policies on the margin} 
\begin{exampleblock}{Limitations to efficient allocation}
 "managerial policies governing the allocation of workers to jobs within the firm, which in principle can be a very important determinant of productivity"
\end{exampleblock}
\begin{itemize}
    \item Optimal managerial decisions may be constrained by a limited pool of sufficiently qualified employed workers in the firm.
    \item For new hires, a good job-worker match may be limited by search frictions.
   \end{itemize}
\end{frame}

\begin{frame} 
\frametitle{ML algorithm I} 
\begin{exampleblock}{Selection on observables}
{"The estimation of a worker’s suitability for a given job is based on the same type of information that
would typically be included in individual resumes available to managers assigning workers to jobs."}
\end{exampleblock}
\begin{itemize}
\item However, managers ability to allocate workers to jobs, to some extent, relies on the their ability to assess the individual's \emph{unobserved} characteristics: temperament, personality etc.
\item You take what you have, but may need to be qualified.
\end{itemize}
\end{frame}

\begin{frame} 
\frametitle{ML algorithm II} 
\begin{exampleblock}{Selection on the most productive firms}
"to include in the learning sample only firms that are consistently more productive, for each size-industry class we (i) estimate a \textbf{model of value added per employee with firm fixed effects and calendar year effects}, (ii) consider the distribution of fixed effects for firms"
\end{exampleblock}
\begin{itemize}
\item Can't see clearly what you would capture with the fixed effect 
\begin{itemize}
\item Average value added per worker, deviations from the grand mean (adjusted for a common trend)? 
\end{itemize}
\item Selection on more capital intensive firms?
\item What about TFP? Productivity throughout mainly to be understood as labor productivity, and more specifically value added per employee?
\end{itemize}
\end{frame}

\begin{frame} 
\frametitle{ML results} 
\begin{itemize}
    \item Measures of JAQ and prob(J) as two extremes. Middle of the road measure? Perhaps taking into account, occupations substitutability.
    \item Would be interesting to see prediction results for different occupations.
    \item Easier to predict professions like engineer, lawyer, accountant compared to e.g. sales worker, shop assistant? 
    \item What I'm after is whether JAQ=1 may select for more highly-skilled occupations?
    \item Since occupations are central to the analysis more basic description would help.
    \item Any information on important predictors? Possible to glimpse into the "black box"?
   \end{itemize}
\end{frame}

\begin{frame}{JAQ and higher earnings}
\begin{exampleblock}{Wage regression (1)}
\end{exampleblock}
\begin{itemize}
\item Considered an AKM type model?
\item Account for the possibility that some firms pay systematically higher wages.
\item Typically finding is a positive link between firm-specific pay policies and firm productivity (including value added per worker and sales per worker) 
\end{itemize}
\emph{\small See Card, D., Cardoso, A. R., Heining, J., \& Kline, P. (2018). Firms and labor market inequality: Evidence and some theory.}
\end{frame}

\begin{frame}{cont.} 
\begin{itemize}
\item Extensions of the AKM model to explicitly account for \underline{match} fixed effects
\end{itemize}
\emph{\small Notably Woodcock, Simon D. "Wage differentials in the presence of unobserved worker, firm, and match heterogeneity." Labour Economics 15.4 (2008): 771-793.}
\end{frame}

\begin{frame}{JAQ and Firm Performance}
\begin{exampleblock}{Firm level regressions}
\end{exampleblock}
\begin{itemize}
\item Is JAQ endogenous? 
\item To the extent JAQ is correlated to labor inputs, firms may anticipate shocks to TFP by adjusting their labor accordingly, and by extension their JAQ. 
\item Within firm variation of JAQ does not significantly affect value added per worker... 
\item If management practices/HR policy is largely fixed, however, you should probably not use firm fixed effect here?  
\end{itemize}
\end{frame}

\begin{frame} 
\frametitle{Other comments} 
\begin{itemize}
    \item No results for the profit equation may reflect the fact that wages adjust accordingly? I.e, gains in value added per worker is offset by increased average wage (zero gross operating surplus).
    \item Consider presenting the responds to the various concerns in a separate section.
    \item Highlight the suitability measure earlier in the paper. It's good.
    \item What about measurement error?
    \item Missed a discussion on the included sample of individuals w.r.t. students, retired, part-time, winsorizing, etc.
    \item No data on sales and assets in LISA?
    \item Financial sector not included in the FEK data?
   \end{itemize}
\end{frame}
\end{document}
